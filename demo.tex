\documentclass{article}
\usepackage[paperheight=16cm, paperwidth=12cm,includehead,nomarginpar,textwidth=10cm,headheight=10mm,]{geometry}
\usepackage{fancyhdr}
\usepackage{hyperref}
\usepackage{datetime2}
\newcommand\docversion{DRAFT; \today\hspace{0.1cm}\DTMcurrenttime}

% if not using overleaf - data is needed separately
\newcommand{\mySubject}{Thesis Book}
\newcommand{\myNameInclTitle}{FirstName LastName}
\newcommand{\myTitle}{Modernizing Developer Workflows: Embracing Containerization in Legacy Software Development}
\newcommand{\PDFdocumentLang}{en-US}


\hypersetup{
    pdftitle={\myTitle},
    pdfsubject={\mySubject},
    pdfauthor={\myNameInclTitle},
    breaklinks, % permits line breaks for long links
    bookmarksnumbered,        % ... and include section numbers
    linktocpage,        % "make page number, not text, be link on TOC ..."
    colorlinks,            % yes ...
    linkcolor=black,        % normal internal links;
    anchorcolor=black,   
    citecolor=black,
    urlcolor=blue,        % quite common
    pdfstartview={Fit},        % "Fit" fits the page to the window
    pdfpagemode=UseOutlines,    % open bookmarks in Acrobat
    plainpages=false,      % avoids duplicate page number problem
    pdflang={\PDFdocumentLang},
    pdfkeywords={\docversion}
}

\begin{document}
% Set the page style to "fancy"...
\pagestyle{fancy}
\title{\myTitle}
\author{\myNameInclTitle}
\date{\docversion}
\fancyhf{} % clear existing header/footer entries
% We don't need to specify the O coordinate
\fancyhead[R]{Hello}
\fancyfoot[L]{\thepage}
\fancyfoot[C]{\docversion}
\maketitle
\section{Introduction}
This is generated from demo.tex. demo.tex can be safely removed.\\\\
The upload to github releases is done automatically only from main.tex/.pdf when you create and publish a release.\\\\
So create a main.tex and you are good to go.
\newpage
\section{Continued...}
\end{document}
% original example from overleaf.com
